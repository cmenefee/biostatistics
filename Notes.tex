\documentclass[]{article}
\usepackage{lmodern}
\usepackage{amssymb,amsmath}
\usepackage{ifxetex,ifluatex}
\usepackage{fixltx2e} % provides \textsubscript
\ifnum 0\ifxetex 1\fi\ifluatex 1\fi=0 % if pdftex
  \usepackage[T1]{fontenc}
  \usepackage[utf8]{inputenc}
\else % if luatex or xelatex
  \ifxetex
    \usepackage{mathspec}
  \else
    \usepackage{fontspec}
  \fi
  \defaultfontfeatures{Ligatures=TeX,Scale=MatchLowercase}
\fi
% use upquote if available, for straight quotes in verbatim environments
\IfFileExists{upquote.sty}{\usepackage{upquote}}{}
% use microtype if available
\IfFileExists{microtype.sty}{%
\usepackage{microtype}
\UseMicrotypeSet[protrusion]{basicmath} % disable protrusion for tt fonts
}{}
\usepackage[margin=1in]{geometry}
\usepackage{hyperref}
\hypersetup{unicode=true,
            pdftitle={R Notebook},
            pdfborder={0 0 0},
            breaklinks=true}
\urlstyle{same}  % don't use monospace font for urls
\usepackage{color}
\usepackage{fancyvrb}
\newcommand{\VerbBar}{|}
\newcommand{\VERB}{\Verb[commandchars=\\\{\}]}
\DefineVerbatimEnvironment{Highlighting}{Verbatim}{commandchars=\\\{\}}
% Add ',fontsize=\small' for more characters per line
\usepackage{framed}
\definecolor{shadecolor}{RGB}{248,248,248}
\newenvironment{Shaded}{\begin{snugshade}}{\end{snugshade}}
\newcommand{\AlertTok}[1]{\textcolor[rgb]{0.94,0.16,0.16}{#1}}
\newcommand{\AnnotationTok}[1]{\textcolor[rgb]{0.56,0.35,0.01}{\textbf{\textit{#1}}}}
\newcommand{\AttributeTok}[1]{\textcolor[rgb]{0.77,0.63,0.00}{#1}}
\newcommand{\BaseNTok}[1]{\textcolor[rgb]{0.00,0.00,0.81}{#1}}
\newcommand{\BuiltInTok}[1]{#1}
\newcommand{\CharTok}[1]{\textcolor[rgb]{0.31,0.60,0.02}{#1}}
\newcommand{\CommentTok}[1]{\textcolor[rgb]{0.56,0.35,0.01}{\textit{#1}}}
\newcommand{\CommentVarTok}[1]{\textcolor[rgb]{0.56,0.35,0.01}{\textbf{\textit{#1}}}}
\newcommand{\ConstantTok}[1]{\textcolor[rgb]{0.00,0.00,0.00}{#1}}
\newcommand{\ControlFlowTok}[1]{\textcolor[rgb]{0.13,0.29,0.53}{\textbf{#1}}}
\newcommand{\DataTypeTok}[1]{\textcolor[rgb]{0.13,0.29,0.53}{#1}}
\newcommand{\DecValTok}[1]{\textcolor[rgb]{0.00,0.00,0.81}{#1}}
\newcommand{\DocumentationTok}[1]{\textcolor[rgb]{0.56,0.35,0.01}{\textbf{\textit{#1}}}}
\newcommand{\ErrorTok}[1]{\textcolor[rgb]{0.64,0.00,0.00}{\textbf{#1}}}
\newcommand{\ExtensionTok}[1]{#1}
\newcommand{\FloatTok}[1]{\textcolor[rgb]{0.00,0.00,0.81}{#1}}
\newcommand{\FunctionTok}[1]{\textcolor[rgb]{0.00,0.00,0.00}{#1}}
\newcommand{\ImportTok}[1]{#1}
\newcommand{\InformationTok}[1]{\textcolor[rgb]{0.56,0.35,0.01}{\textbf{\textit{#1}}}}
\newcommand{\KeywordTok}[1]{\textcolor[rgb]{0.13,0.29,0.53}{\textbf{#1}}}
\newcommand{\NormalTok}[1]{#1}
\newcommand{\OperatorTok}[1]{\textcolor[rgb]{0.81,0.36,0.00}{\textbf{#1}}}
\newcommand{\OtherTok}[1]{\textcolor[rgb]{0.56,0.35,0.01}{#1}}
\newcommand{\PreprocessorTok}[1]{\textcolor[rgb]{0.56,0.35,0.01}{\textit{#1}}}
\newcommand{\RegionMarkerTok}[1]{#1}
\newcommand{\SpecialCharTok}[1]{\textcolor[rgb]{0.00,0.00,0.00}{#1}}
\newcommand{\SpecialStringTok}[1]{\textcolor[rgb]{0.31,0.60,0.02}{#1}}
\newcommand{\StringTok}[1]{\textcolor[rgb]{0.31,0.60,0.02}{#1}}
\newcommand{\VariableTok}[1]{\textcolor[rgb]{0.00,0.00,0.00}{#1}}
\newcommand{\VerbatimStringTok}[1]{\textcolor[rgb]{0.31,0.60,0.02}{#1}}
\newcommand{\WarningTok}[1]{\textcolor[rgb]{0.56,0.35,0.01}{\textbf{\textit{#1}}}}
\usepackage{graphicx,grffile}
\makeatletter
\def\maxwidth{\ifdim\Gin@nat@width>\linewidth\linewidth\else\Gin@nat@width\fi}
\def\maxheight{\ifdim\Gin@nat@height>\textheight\textheight\else\Gin@nat@height\fi}
\makeatother
% Scale images if necessary, so that they will not overflow the page
% margins by default, and it is still possible to overwrite the defaults
% using explicit options in \includegraphics[width, height, ...]{}
\setkeys{Gin}{width=\maxwidth,height=\maxheight,keepaspectratio}
\IfFileExists{parskip.sty}{%
\usepackage{parskip}
}{% else
\setlength{\parindent}{0pt}
\setlength{\parskip}{6pt plus 2pt minus 1pt}
}
\setlength{\emergencystretch}{3em}  % prevent overfull lines
\providecommand{\tightlist}{%
  \setlength{\itemsep}{0pt}\setlength{\parskip}{0pt}}
\setcounter{secnumdepth}{0}
% Redefines (sub)paragraphs to behave more like sections
\ifx\paragraph\undefined\else
\let\oldparagraph\paragraph
\renewcommand{\paragraph}[1]{\oldparagraph{#1}\mbox{}}
\fi
\ifx\subparagraph\undefined\else
\let\oldsubparagraph\subparagraph
\renewcommand{\subparagraph}[1]{\oldsubparagraph{#1}\mbox{}}
\fi

%%% Use protect on footnotes to avoid problems with footnotes in titles
\let\rmarkdownfootnote\footnote%
\def\footnote{\protect\rmarkdownfootnote}

%%% Change title format to be more compact
\usepackage{titling}

% Create subtitle command for use in maketitle
\providecommand{\subtitle}[1]{
  \posttitle{
    \begin{center}\large#1\end{center}
    }
}

\setlength{\droptitle}{-2em}

  \title{R Notebook}
    \pretitle{\vspace{\droptitle}\centering\huge}
  \posttitle{\par}
    \author{}
    \preauthor{}\postauthor{}
    \date{}
    \predate{}\postdate{}
  

\begin{document}
\maketitle

\hypertarget{project-details}{%
\section{Project Details}\label{project-details}}

\hypertarget{the-book}{%
\subsection{The Book}\label{the-book}}

Methods in Biostatistics with R Ciprian Crainiceanu, Brian Caffo, John
Muschelli

\begin{itemize}
\tightlist
\item
  Intro Video: \url{https://www.youtube.com/watch?v=MG1lZhixbS4}
\item
  Site: \url{https://leanpub.com/biostatmethods}
\end{itemize}

\hypertarget{the-data}{%
\subsection{The Data}\label{the-data}}

From the Book

\begin{itemize}
\tightlist
\item
  \url{https://github.com/muschellij2/biostatmethods}
\end{itemize}

\hypertarget{chapter-1.-introduction}{%
\section{Chapter 1. Introduction}\label{chapter-1.-introduction}}

``\ldots{}.We provide a modern look at introductory biostatistical
concepts and associated computational tools, reflecting the latest
developments in computation and vi- sualization using the R language
environment (R Core Team 2016). The idea is to offer a complete, online,
live book that evolves with the newest developments and is continuously
enriched by additional concepts\ldots{}''

1.1 Biostatistics

``\ldots{}.Biostatistics is the theory and methodology for the
acquisition and use of quantitative evidence in biomedical research.
Biostatisticians develop innovative designs and analytic methods
targeted at increasing available information, improving the relevance
and validity of statistical analyses, making best use of available
information and communicating relevant uncertainties\ldots{}.''

``\ldots{}.Biostatistics is a challenging subject to master if one is
interested in understanding both the underlying concepts and their
appropriate implementation. \ldots{}.Biostatistics can be viewed as a
powerful and practical philosophy of science, in which the scientific
hypothesis, the experiment, the data, the model, and the associated
inference form the basis of scientific progress. \ldots{}.Biostatistics
is hard because it is focused on solving difficult problems using simple
approaches and simplicity is difficult to achieve in complex scenarios.
\ldots{}.Biostatistics is not hard in the usual sense that it requires
either deep mathematical concepts or committing large books to
memory\ldots{}.''

``\ldots{}.The difficulty of biostatistics rests in the different
philosophy, the different way of thinking, and the different set of
skills required to process human communication of scientific problems
and their translation into well-defined problems that can be solved with
data. Once a scientific promlem is defined, the biostatistics philosophy
is to try to solve it using the simplest possible approaches that are
not too simplistic. Thus, parsimony and inductive reasoning are
fundamental concepts in biostatistics\ldots{}.''

``\ldots{}.biostatistics starts with the current accepted state of
knowledge (the collection of null hypotheses) and uses data to
inductively refute or reinforce parts of the knowledge or generate new
potential knowledge (new null hypotheses). In the biostatistical
philosophy there is no truth, just the state of current
knowledge\ldots{}.''

``\ldots{}.Biostatistics requires: (1) a tight coupling of the
biostatistical methods with ethical and scientific goals of research;
(2) an emphasis on the scientific interpretation of statistical evidence
to impact policy; and (3) a detailed acknowledgment of assumptions and a
comprehensive evaluation of the robustness of conclusions to these
assumptions\ldots{}.''

\hypertarget{chapter-2.-introduction-to-r}{%
\section{Chapter 2. Introduction to
R}\label{chapter-2.-introduction-to-r}}

\hypertarget{objects-in-r}{%
\subsection{2.4 Objects In R}\label{objects-in-r}}

``\ldots{}.Almost everything is an ``object'' or
``variable.''\ldots{}.most things are relevant only with respect to a
data set\ldots{}."

\hypertarget{assignment}{%
\subsection{2.5 Assignment}\label{assignment}}

``\ldots{}.There is also a ``forward assignment'' operator that exists,
but this is rarely used and we do not recommend using it\ldots{}." Oh
no.

\hypertarget{data-types-in-r}{%
\subsection{2.6 Data types In R}\label{data-types-in-r}}

``\ldots{}.The three most fundamental: numeric (numbers), character
(strings or ``words''), and logicals (TRUE or FALSE).

``\ldots{}.One additional data type: factors. Factors are categorical
data types. The categories of a factor are called the ``levels'' of that
factor. The levels of factors usually contain human-readable levels
(e.g., ``Male'' and ``Female''). Factors become increasingly useful when
we fit models, as one can define the baseline category by the levels of
the factor, which affects the interpretation of coefficients. Some refer
to vectors of length 1 as a ``scalar''. In other languages, there is a
distinction between scalars and vectors, but not in R. So you can think
of a vector as a 1-dimensional object that can have anywhere from 0 to a
large number of elements\ldots{}."

\hypertarget{more-than-one-number-vectors-in-r}{%
\subsection{2.7.1 More than one number: vectors In
R}\label{more-than-one-number-vectors-in-r}}

``\ldots{}.Anything having one dimension is generally referred to as a
``vector''. A vector has a specified length. Technically, a vector may
have length 0, which means nothing is in it. But usually vectors have
one or multiple objects in them. Those objects all must have the same
data type (e.g., all numbers or all characters or all
logicals)\ldots{}."

\hypertarget{sequences}{%
\subsection{2.7.3 Sequences}\label{sequences}}

\begin{Shaded}
\begin{Highlighting}[]
\DecValTok{1}\OperatorTok{:}\DecValTok{5}
\end{Highlighting}
\end{Shaded}

\begin{verbatim}
## [1] 1 2 3 4 5
\end{verbatim}

\begin{Shaded}
\begin{Highlighting}[]
\DecValTok{-5}\OperatorTok{:}\DecValTok{1}
\end{Highlighting}
\end{Shaded}

\begin{verbatim}
## [1] -5 -4 -3 -2 -1  0  1
\end{verbatim}

\begin{Shaded}
\begin{Highlighting}[]
\OperatorTok{-}\NormalTok{(}\DecValTok{5}\OperatorTok{:}\DecValTok{1}\NormalTok{)}
\end{Highlighting}
\end{Shaded}

\begin{verbatim}
## [1] -5 -4 -3 -2 -1
\end{verbatim}

\begin{Shaded}
\begin{Highlighting}[]
\KeywordTok{seq}\NormalTok{(}\DecValTok{1}\NormalTok{,}\DecValTok{5}\NormalTok{)}
\end{Highlighting}
\end{Shaded}

\begin{verbatim}
## [1] 1 2 3 4 5
\end{verbatim}

\begin{Shaded}
\begin{Highlighting}[]
\KeywordTok{seq}\NormalTok{(}\DecValTok{1}\NormalTok{,}\DecValTok{7}\NormalTok{, }\DataTypeTok{by =} \FloatTok{0.4}\NormalTok{)}
\end{Highlighting}
\end{Shaded}

\begin{verbatim}
##  [1] 1.0 1.4 1.8 2.2 2.6 3.0 3.4 3.8 4.2 4.6 5.0 5.4 5.8 6.2 6.6 7.0
\end{verbatim}

\begin{Shaded}
\begin{Highlighting}[]
\KeywordTok{seq}\NormalTok{(}\DataTypeTok{by =} \FloatTok{0.4}\NormalTok{, }\DataTypeTok{to =} \DecValTok{7}\NormalTok{, }\DataTypeTok{from =} \DecValTok{1}\NormalTok{)}
\end{Highlighting}
\end{Shaded}

\begin{verbatim}
##  [1] 1.0 1.4 1.8 2.2 2.6 3.0 3.4 3.8 4.2 4.6 5.0 5.4 5.8 6.2 6.6 7.0
\end{verbatim}

\hypertarget{operations-on-numeric-vectors}{%
\subsection{2.7.4 Operations on numeric
vectors}\label{operations-on-numeric-vectors}}

\begin{Shaded}
\begin{Highlighting}[]
\NormalTok{x_length_}\DecValTok{2}\NormalTok{ =}\StringTok{ }\KeywordTok{c}\NormalTok{(}\DecValTok{1}\NormalTok{, }\DecValTok{2}\NormalTok{) }
\NormalTok{x_length_}\DecValTok{2} \OperatorTok{*}\StringTok{ }\DecValTok{3}
\end{Highlighting}
\end{Shaded}

\begin{verbatim}
## [1] 3 6
\end{verbatim}

\begin{Shaded}
\begin{Highlighting}[]
\KeywordTok{sqrt}\NormalTok{(x_length_}\DecValTok{2}\NormalTok{)}
\end{Highlighting}
\end{Shaded}

\begin{verbatim}
## [1] 1.000000 1.414214
\end{verbatim}

\hypertarget{some-standard-operations-on-vectors}{%
\subsection{2.7.5 Some standard operations on
vectors}\label{some-standard-operations-on-vectors}}

\begin{Shaded}
\begin{Highlighting}[]
\NormalTok{x_length_}\DecValTok{3}\NormalTok{ =}\StringTok{ }\KeywordTok{c}\NormalTok{(}\DecValTok{10}\NormalTok{, }\DecValTok{20}\NormalTok{, }\DecValTok{30}\NormalTok{)}
\KeywordTok{length}\NormalTok{(x_length_}\DecValTok{3}\NormalTok{)}
\end{Highlighting}
\end{Shaded}

\begin{verbatim}
## [1] 3
\end{verbatim}

\begin{Shaded}
\begin{Highlighting}[]
\NormalTok{x =}\StringTok{ }\KeywordTok{c}\NormalTok{(}\DecValTok{2}\NormalTok{, }\DecValTok{1}\NormalTok{, }\DecValTok{1}\NormalTok{, }\DecValTok{4}\NormalTok{)}
\KeywordTok{sort}\NormalTok{(}\KeywordTok{unique}\NormalTok{(x))}
\end{Highlighting}
\end{Shaded}

\begin{verbatim}
## [1] 1 2 4
\end{verbatim}

\begin{Shaded}
\begin{Highlighting}[]
\KeywordTok{sample}\NormalTok{(x)}
\end{Highlighting}
\end{Shaded}

\begin{verbatim}
## [1] 4 1 2 1
\end{verbatim}

\begin{Shaded}
\begin{Highlighting}[]
\KeywordTok{sample}\NormalTok{(x, }\DataTypeTok{size =} \DecValTok{2}\NormalTok{)}
\end{Highlighting}
\end{Shaded}

\begin{verbatim}
## [1] 4 2
\end{verbatim}

\begin{Shaded}
\begin{Highlighting}[]
\CommentTok{# sample(x, size = 10) will error }
\KeywordTok{sample}\NormalTok{(x, }\DataTypeTok{size =} \DecValTok{10}\NormalTok{, }\DataTypeTok{replace =} \OtherTok{TRUE}\NormalTok{)}
\end{Highlighting}
\end{Shaded}

\begin{verbatim}
##  [1] 1 2 4 1 4 2 4 2 1 1
\end{verbatim}

\hypertarget{data-type-coercion}{%
\subsection{2.7.7 Data type coercion}\label{data-type-coercion}}

``\ldots{}.Vectors can only have one data type in them\ldots{}.''

``\ldots{}.Logicals can be coerced to numerics (0 and 1 values) and
coerced to characters (''TRUE" and ``FALSE'', note the quotes). Numerics
can be coerced into characters (5.6 changes to ``5.6'')\ldots{}."

You can determine the data type of an object with the typeof() function:

\begin{Shaded}
\begin{Highlighting}[]
\KeywordTok{typeof}\NormalTok{(}\KeywordTok{c}\NormalTok{(}\FloatTok{5.2}\NormalTok{, }\OtherTok{TRUE}\NormalTok{, }\OtherTok{FALSE}\NormalTok{))}
\end{Highlighting}
\end{Shaded}

\begin{verbatim}
## [1] "double"
\end{verbatim}

\hypertarget{more-than-one-dimension-matrices}{%
\subsection{2.7.8 More than one dimension:
Matrices}\label{more-than-one-dimension-matrices}}

``\ldots{}.A matrix is like a vector, but it has two dimensions, rows
and columns. Like vectors, matrices can contain only one data
type\ldots{}.'' So I imagine a vector is just `A row' and a matrix is `A
row and a column', or something like that

\begin{Shaded}
\begin{Highlighting}[]
\KeywordTok{matrix}\NormalTok{(}\DecValTok{1}\NormalTok{)}
\end{Highlighting}
\end{Shaded}

\begin{verbatim}
##      [,1]
## [1,]    1
\end{verbatim}

\begin{Shaded}
\begin{Highlighting}[]
\KeywordTok{matrix}\NormalTok{(}\DecValTok{1}\OperatorTok{:}\DecValTok{6}\NormalTok{, }\DataTypeTok{nrow =} \DecValTok{2}\NormalTok{, }\DataTypeTok{ncol =} \DecValTok{3}\NormalTok{) }
\end{Highlighting}
\end{Shaded}

\begin{verbatim}
##      [,1] [,2] [,3]
## [1,]    1    3    5
## [2,]    2    4    6
\end{verbatim}

\begin{Shaded}
\begin{Highlighting}[]
\KeywordTok{matrix}\NormalTok{(}\DecValTok{1}\OperatorTok{:}\DecValTok{6}\NormalTok{, }\DataTypeTok{nrow =} \DecValTok{2}\NormalTok{, }\DataTypeTok{ncol =} \DecValTok{3}\NormalTok{, }\DataTypeTok{byrow =} \OtherTok{TRUE}\NormalTok{) }\CommentTok{# filled by row}
\end{Highlighting}
\end{Shaded}

\begin{verbatim}
##      [,1] [,2] [,3]
## [1,]    1    2    3
## [2,]    4    5    6
\end{verbatim}

\hypertarget{more-than-one-dimension-data.frames}{%
\subsection{2.7.9 More than one dimension:
data.frames}\label{more-than-one-dimension-data.frames}}

``\ldots{}.A data.frame is like a matrix in that it has rows and
columns. A data.frame can have columns of different data
types\ldots{}.''

\ldots{}.Note, a default on the data.frame() function is
stringsAsFactors = TRUE, which indicates that all strings should be
converted to factors. Many times, you want to clean these columns and
then convert them to factors directly. So we will set that option to
FALSE\ldots{}."

\begin{Shaded}
\begin{Highlighting}[]
\NormalTok{df =}\StringTok{ }\KeywordTok{data.frame}\NormalTok{(}\DataTypeTok{age =} \KeywordTok{c}\NormalTok{(}\DecValTok{25}\NormalTok{, }\DecValTok{30}\NormalTok{, }\DecValTok{32}\NormalTok{, }\DecValTok{42}\NormalTok{), }
                \DataTypeTok{handed =} \KeywordTok{c}\NormalTok{(}\StringTok{"left"}\NormalTok{, }\StringTok{"right"}\NormalTok{, }\StringTok{"ambidextrous"}\NormalTok{, }\StringTok{"left"}\NormalTok{), }
                \DataTypeTok{stringsAsFactors =} \OtherTok{FALSE}\NormalTok{)}
\end{Highlighting}
\end{Shaded}

\hypertarget{dimensions-of-an-object}{%
\subsection{2.7.10 Dimensions of an
object}\label{dimensions-of-an-object}}

There are functions that get attributes about a matrix/data.frame that
are useful for analysis.

The dim function returns the dimensions (rows and columns, respectively)
of a data set:

\begin{Shaded}
\begin{Highlighting}[]
\NormalTok{df =}\StringTok{ }\KeywordTok{data.frame}\NormalTok{(}\DataTypeTok{age =} \KeywordTok{c}\NormalTok{(}\DecValTok{25}\NormalTok{, }\DecValTok{30}\NormalTok{, }\DecValTok{32}\NormalTok{, }\DecValTok{42}\NormalTok{), }
                \DataTypeTok{handed =} \KeywordTok{c}\NormalTok{(}\StringTok{"left"}\NormalTok{, }\StringTok{"right"}\NormalTok{, }\StringTok{"ambidextrous"}\NormalTok{, }\StringTok{"left"}\NormalTok{), }
                \DataTypeTok{stringsAsFactors =} \OtherTok{FALSE}\NormalTok{)}
\KeywordTok{dim}\NormalTok{(df) }
\end{Highlighting}
\end{Shaded}

\begin{verbatim}
## [1] 4 2
\end{verbatim}

Number of rows or columns

\begin{Shaded}
\begin{Highlighting}[]
\NormalTok{df =}\StringTok{ }\KeywordTok{data.frame}\NormalTok{(}\DataTypeTok{age =} \KeywordTok{c}\NormalTok{(}\DecValTok{25}\NormalTok{, }\DecValTok{30}\NormalTok{, }\DecValTok{32}\NormalTok{, }\DecValTok{42}\NormalTok{), }
                \DataTypeTok{handed =} \KeywordTok{c}\NormalTok{(}\StringTok{"left"}\NormalTok{, }\StringTok{"right"}\NormalTok{, }\StringTok{"ambidextrous"}\NormalTok{, }\StringTok{"left"}\NormalTok{), }
                \DataTypeTok{stringsAsFactors =} \OtherTok{FALSE}\NormalTok{)}
\KeywordTok{nrow}\NormalTok{(df)}
\end{Highlighting}
\end{Shaded}

\begin{verbatim}
## [1] 4
\end{verbatim}

\begin{Shaded}
\begin{Highlighting}[]
\KeywordTok{ncol}\NormalTok{(df)}
\end{Highlighting}
\end{Shaded}

\begin{verbatim}
## [1] 2
\end{verbatim}

\hypertarget{logical-operations}{%
\subsection{2.8 Logical operations}\label{logical-operations}}

``\ldots{}.Perform some logical tests that return logical data.
greater/less than (\textgreater{}, \textless{}) greater/less than or
equal to (\textgreater{}=, \textless{}=) equal to (== ``double equals'')
and not equal to (!=) These are called ``relational operators'' or
``comparison operators''

There are also the standard ``logical operators'': not/negate (!), AND
(\&), and OR (\textbar{})\ldots{}."

\begin{Shaded}
\begin{Highlighting}[]
\NormalTok{x =}\StringTok{ }\KeywordTok{c}\NormalTok{(}\DecValTok{1}\NormalTok{, }\DecValTok{3}\NormalTok{, }\DecValTok{4}\NormalTok{, }\DecValTok{5}\NormalTok{) }
\NormalTok{x }\OperatorTok{<}\StringTok{ }\DecValTok{4}
\end{Highlighting}
\end{Shaded}

\begin{verbatim}
## [1]  TRUE  TRUE FALSE FALSE
\end{verbatim}

\begin{Shaded}
\begin{Highlighting}[]
\NormalTok{x }\OperatorTok{>}\StringTok{ }\DecValTok{3}
\end{Highlighting}
\end{Shaded}

\begin{verbatim}
## [1] FALSE FALSE  TRUE  TRUE
\end{verbatim}

\begin{Shaded}
\begin{Highlighting}[]
\NormalTok{x }\OperatorTok{>=}\StringTok{ }\DecValTok{3}
\end{Highlighting}
\end{Shaded}

\begin{verbatim}
## [1] FALSE  TRUE  TRUE  TRUE
\end{verbatim}

\begin{Shaded}
\begin{Highlighting}[]
\NormalTok{x }\OperatorTok{==}\StringTok{ }\DecValTok{3}
\end{Highlighting}
\end{Shaded}

\begin{verbatim}
## [1] FALSE  TRUE FALSE FALSE
\end{verbatim}

\begin{Shaded}
\begin{Highlighting}[]
\NormalTok{x }\OperatorTok{!=}\StringTok{ }\DecValTok{3}
\end{Highlighting}
\end{Shaded}

\begin{verbatim}
## [1]  TRUE FALSE  TRUE  TRUE
\end{verbatim}

\begin{Shaded}
\begin{Highlighting}[]
\NormalTok{x =}\StringTok{ }\KeywordTok{c}\NormalTok{(}\DecValTok{1}\NormalTok{, }\DecValTok{3}\NormalTok{, }\DecValTok{4}\NormalTok{, }\DecValTok{5}\NormalTok{) }
\NormalTok{x }\OperatorTok{>}\StringTok{ }\DecValTok{3} \OperatorTok{|}\StringTok{ }\NormalTok{x }\OperatorTok{==}\StringTok{ }\DecValTok{1} 
\end{Highlighting}
\end{Shaded}

\begin{verbatim}
## [1]  TRUE FALSE  TRUE  TRUE
\end{verbatim}

\begin{Shaded}
\begin{Highlighting}[]
\OperatorTok{!}\NormalTok{(x }\OperatorTok{>}\StringTok{ }\DecValTok{3} \OperatorTok{|}\StringTok{ }\NormalTok{x }\OperatorTok{==}\StringTok{ }\DecValTok{1}\NormalTok{) }\CommentTok{# negates the statement - turns TRUE to FALSE [1] FALSE TRUE FALSE FALSE }
\end{Highlighting}
\end{Shaded}

\begin{verbatim}
## [1] FALSE  TRUE FALSE FALSE
\end{verbatim}

\begin{Shaded}
\begin{Highlighting}[]
\NormalTok{x }\OperatorTok{!=}\StringTok{ }\DecValTok{3} \OperatorTok{&}\StringTok{ }\NormalTok{x }\OperatorTok{!=}\StringTok{ }\DecValTok{5}
\end{Highlighting}
\end{Shaded}

\begin{verbatim}
## [1]  TRUE FALSE  TRUE FALSE
\end{verbatim}

\begin{Shaded}
\begin{Highlighting}[]
\NormalTok{x }\OperatorTok{==}\StringTok{ }\DecValTok{3} \OperatorTok{|}\StringTok{ }\NormalTok{x }\OperatorTok{==}\StringTok{ }\DecValTok{4}
\end{Highlighting}
\end{Shaded}

\begin{verbatim}
## [1] FALSE  TRUE  TRUE FALSE
\end{verbatim}

\hypertarget{the-in-operator}{%
\subsection{2.8.1 The \%in\% Operator}\label{the-in-operator}}

``\ldots{}.Many times, we want to keep or exclude based on a number of
values\ldots{}.''

\begin{Shaded}
\begin{Highlighting}[]
\NormalTok{x =}\StringTok{ }\KeywordTok{c}\NormalTok{(}\DecValTok{1}\NormalTok{, }\DecValTok{3}\NormalTok{, }\DecValTok{4}\NormalTok{, }\DecValTok{5}\NormalTok{) }
\NormalTok{x }\OperatorTok\StringTok{ }\KeywordTok{c}\NormalTok{(}\DecValTok{3}\NormalTok{,}\DecValTok{4}\NormalTok{)}
\end{Highlighting}
\end{Shaded}

\begin{verbatim}
## [1] FALSE  TRUE  TRUE FALSE
\end{verbatim}

This feels like an ``is in?'' or may even an ``is?'' in that it asks:
``are any of these values in the vector?''.

\hypertarget{subsetting-2.9.1-subsetting-vectors}{%
\subsection{2.9 Subsetting 2.9.1 Subsetting
vectors}\label{subsetting-2.9.1-subsetting-vectors}}

\begin{Shaded}
\begin{Highlighting}[]
\NormalTok{x =}\StringTok{ }\KeywordTok{c}\NormalTok{(}\DecValTok{1}\NormalTok{, }\DecValTok{3}\NormalTok{, }\DecValTok{4}\NormalTok{, }\DecValTok{5}\NormalTok{)}
\NormalTok{x[}\DecValTok{4}\NormalTok{]}
\end{Highlighting}
\end{Shaded}

\begin{verbatim}
## [1] 5
\end{verbatim}

\begin{Shaded}
\begin{Highlighting}[]
\NormalTok{x }\OperatorTok{>}\StringTok{ }\DecValTok{2}
\end{Highlighting}
\end{Shaded}

\begin{verbatim}
## [1] FALSE  TRUE  TRUE  TRUE
\end{verbatim}

\begin{Shaded}
\begin{Highlighting}[]
\NormalTok{x[ x }\OperatorTok{>}\StringTok{ }\DecValTok{2}\NormalTok{ ]}
\end{Highlighting}
\end{Shaded}

\begin{verbatim}
## [1] 3 4 5
\end{verbatim}

\hypertarget{subsetting-matrices}{%
\subsection{2.9.2 Subsetting matrices}\label{subsetting-matrices}}

``\ldots{}.the syntax is {[}row subset, column subset{]}\ldots{}.''

\begin{Shaded}
\begin{Highlighting}[]
\NormalTok{mat =}\StringTok{ }\KeywordTok{matrix}\NormalTok{(}\DecValTok{1}\OperatorTok{:}\DecValTok{6}\NormalTok{, }\DataTypeTok{nrow =} \DecValTok{2}\NormalTok{)}
\NormalTok{mat[}\DecValTok{1}\OperatorTok{:}\DecValTok{2}\NormalTok{, }\DecValTok{1}\NormalTok{]}
\end{Highlighting}
\end{Shaded}

\begin{verbatim}
## [1] 1 2
\end{verbatim}

\begin{Shaded}
\begin{Highlighting}[]
\NormalTok{mat[}\DecValTok{1}\NormalTok{, }\DecValTok{1}\OperatorTok{:}\DecValTok{2}\NormalTok{]}
\end{Highlighting}
\end{Shaded}

\begin{verbatim}
## [1] 1 3
\end{verbatim}

\begin{Shaded}
\begin{Highlighting}[]
\NormalTok{mat[}\DecValTok{1}\NormalTok{, }\KeywordTok{c}\NormalTok{(}\OtherTok{FALSE}\NormalTok{, }\OtherTok{TRUE}\NormalTok{, }\OtherTok{FALSE}\NormalTok{)]}
\end{Highlighting}
\end{Shaded}

\begin{verbatim}
## [1] 3
\end{verbatim}

\begin{Shaded}
\begin{Highlighting}[]
\NormalTok{mat[, }\KeywordTok{c}\NormalTok{(}\OtherTok{FALSE}\NormalTok{, }\DecValTok{3}\NormalTok{, }\OtherTok{TRUE}\NormalTok{)]}
\end{Highlighting}
\end{Shaded}

\begin{verbatim}
##      [,1] [,2]
## [1,]    5    1
## [2,]    6    2
\end{verbatim}

\begin{Shaded}
\begin{Highlighting}[]
\NormalTok{mat[}\DecValTok{1}\OperatorTok{:}\DecValTok{2}\NormalTok{, ]}
\end{Highlighting}
\end{Shaded}

\begin{verbatim}
##      [,1] [,2] [,3]
## [1,]    1    3    5
## [2,]    2    4    6
\end{verbatim}

\begin{Shaded}
\begin{Highlighting}[]
\NormalTok{mat[, }\DecValTok{2}\OperatorTok{:}\DecValTok{3}\NormalTok{]}
\end{Highlighting}
\end{Shaded}

\begin{verbatim}
##      [,1] [,2]
## [1,]    3    5
## [2,]    4    6
\end{verbatim}

\begin{Shaded}
\begin{Highlighting}[]
\NormalTok{mat[, }\KeywordTok{c}\NormalTok{(}\OtherTok{FALSE}\NormalTok{, }\OtherTok{TRUE}\NormalTok{, }\OtherTok{FALSE}\NormalTok{)]}
\end{Highlighting}
\end{Shaded}

\begin{verbatim}
## [1] 3 4
\end{verbatim}

``\ldots{}.You can ensure that the result is a matrix, even if the
result is one-dimensional using the drop = FALSE argument\ldots{}.''

\begin{Shaded}
\begin{Highlighting}[]
\NormalTok{mat[,}\DecValTok{1}\NormalTok{, drop =}\StringTok{ }\OtherTok{FALSE}\NormalTok{]}
\end{Highlighting}
\end{Shaded}

\begin{verbatim}
##      [,1]
## [1,]    1
## [2,]    2
\end{verbatim}

\hypertarget{tibbles}{%
\subsection{2.12.1 Tibbles}\label{tibbles}}

Implemented in the tibble package. It is like a data.frame

\begin{itemize}
\tightlist
\item
  Never coerces inputs (i.e.~strings stay as strings!).
\item
  Never adds row.names.
\item
  Never changes column names.
\item
  Only recycles length 1 inputs. (no wraparound)
\item
  Automatically adds column names.
\end{itemize}

\begin{Shaded}
\begin{Highlighting}[]
\NormalTok{tbl =}\StringTok{ }\NormalTok{tibble}\OperatorTok{::}\KeywordTok{as_tibble}\NormalTok{(df)}
\end{Highlighting}
\end{Shaded}

\hypertarget{problems}{%
\subsection{2.13 Problems}\label{problems}}

\hypertarget{problem-1.-consider-the-vector-num---c1.5-2-8.4}{%
\subsubsection{Problem 1. Consider the vector: num \textless{}-
c(``1.5'', ``2'',
``8.4'')}\label{problem-1.-consider-the-vector-num---c1.5-2-8.4}}

\begin{enumerate}
\def\labelenumi{\alph{enumi}.}
\tightlist
\item
  Convert num into a numeric vector using as.numeric
\end{enumerate}

\begin{Shaded}
\begin{Highlighting}[]
\NormalTok{num <-}\StringTok{ }\KeywordTok{c}\NormalTok{(}\StringTok{"1.5"}\NormalTok{, }\StringTok{"2"}\NormalTok{, }\StringTok{"8.4"}\NormalTok{)}
\KeywordTok{as.numeric}\NormalTok{(num)}
\end{Highlighting}
\end{Shaded}

\begin{verbatim}
## [1] 1.5 2.0 8.4
\end{verbatim}

\begin{enumerate}
\def\labelenumi{\alph{enumi}.}
\setcounter{enumi}{1}
\tightlist
\item
  Convert num into a factor using factor, calling it num\_fac.
\end{enumerate}

\begin{Shaded}
\begin{Highlighting}[]
\NormalTok{num <-}\StringTok{ }\KeywordTok{c}\NormalTok{(}\StringTok{"1.5"}\NormalTok{, }\StringTok{"2"}\NormalTok{, }\StringTok{"8.4"}\NormalTok{)}
\NormalTok{num_fac <-}\StringTok{ }\KeywordTok{factor}\NormalTok{(num)}
\end{Highlighting}
\end{Shaded}

\begin{enumerate}
\def\labelenumi{\alph{enumi}.}
\setcounter{enumi}{2}
\tightlist
\item
  Convert num\_fac into a numeric using as.numeric.
\end{enumerate}

\begin{Shaded}
\begin{Highlighting}[]
\KeywordTok{as.numeric}\NormalTok{(num_fac)}
\end{Highlighting}
\end{Shaded}

\begin{verbatim}
## [1] 1 2 3
\end{verbatim}

\begin{enumerate}
\def\labelenumi{\alph{enumi}.}
\setcounter{enumi}{3}
\tightlist
\item
  Convert num\_fac into a numeric using as.numeric(as.character())
\end{enumerate}

\begin{Shaded}
\begin{Highlighting}[]
\KeywordTok{as.numeric}\NormalTok{(}\KeywordTok{as.character}\NormalTok{(num_fac))}
\end{Highlighting}
\end{Shaded}

\begin{verbatim}
## [1] 1.5 2.0 8.4
\end{verbatim}

\hypertarget{problem-2.-consider-the-vector-num---c0-1-1-0}{%
\subsubsection{Problem 2. Consider the vector: num \textless{}- c(0, 1,
1, 0)}\label{problem-2.-consider-the-vector-num---c0-1-1-0}}

\begin{enumerate}
\def\labelenumi{\alph{enumi}.}
\tightlist
\item
  Append TRUE to the num vector using c.
\end{enumerate}

\begin{Shaded}
\begin{Highlighting}[]
\NormalTok{num <-}\StringTok{ }\KeywordTok{c}\NormalTok{(}\DecValTok{0}\NormalTok{, }\DecValTok{1}\NormalTok{, }\DecValTok{1}\NormalTok{, }\DecValTok{0}\NormalTok{) }
\NormalTok{num <-}\StringTok{ }\KeywordTok{c}\NormalTok{(num, }\OtherTok{TRUE}\NormalTok{)}
\NormalTok{num}
\end{Highlighting}
\end{Shaded}

\begin{verbatim}
## [1] 0 1 1 0 1
\end{verbatim}

\begin{enumerate}
\def\labelenumi{\alph{enumi}.}
\setcounter{enumi}{1}
\tightlist
\item
  Check the class of num using class.
\end{enumerate}

\begin{Shaded}
\begin{Highlighting}[]
\NormalTok{num <-}\StringTok{ }\KeywordTok{c}\NormalTok{(}\DecValTok{0}\NormalTok{, }\DecValTok{1}\NormalTok{, }\DecValTok{1}\NormalTok{, }\DecValTok{0}\NormalTok{) }
\KeywordTok{class}\NormalTok{(num)}
\end{Highlighting}
\end{Shaded}

\begin{verbatim}
## [1] "numeric"
\end{verbatim}

\begin{enumerate}
\def\labelenumi{\alph{enumi}.}
\setcounter{enumi}{2}
\tightlist
\item
  Convert num into a logical vector using as.logical.
\end{enumerate}

\begin{Shaded}
\begin{Highlighting}[]
\NormalTok{num <-}\StringTok{ }\KeywordTok{c}\NormalTok{(}\DecValTok{0}\NormalTok{, }\DecValTok{1}\NormalTok{, }\DecValTok{1}\NormalTok{, }\DecValTok{0}\NormalTok{) }
\KeywordTok{as.logical}\NormalTok{(num)}
\end{Highlighting}
\end{Shaded}

\begin{verbatim}
## [1] FALSE  TRUE  TRUE FALSE
\end{verbatim}

\begin{enumerate}
\def\labelenumi{\alph{enumi}.}
\setcounter{enumi}{3}
\tightlist
\item
  Convert num into a logical vector where the result is TRUE if num is
  0, using the == operator.
\end{enumerate}

\begin{Shaded}
\begin{Highlighting}[]
\NormalTok{num <-}\StringTok{ }\KeywordTok{c}\NormalTok{(}\DecValTok{0}\NormalTok{, }\DecValTok{1}\NormalTok{, }\DecValTok{1}\NormalTok{, }\DecValTok{0}\NormalTok{) }
\KeywordTok{as.logical}\NormalTok{(num }\OperatorTok{==}\StringTok{ }\DecValTok{0}\NormalTok{)}
\end{Highlighting}
\end{Shaded}

\begin{verbatim}
## [1]  TRUE FALSE FALSE  TRUE
\end{verbatim}

\hypertarget{problem-3.-consider-the-data-set-data_bmi-from-above.-perform-the-following-operations}{%
\subsubsection{Problem 3. Consider the data set data\_bmi from above.
Perform the following
operations:}\label{problem-3.-consider-the-data-set-data_bmi-from-above.-perform-the-following-operations}}

\begin{enumerate}
\def\labelenumi{\alph{enumi}.}
\tightlist
\item
  Extract the column nams of data\_bmi using the colnames() function.
\end{enumerate}

\begin{Shaded}
\begin{Highlighting}[]
\NormalTok{file_name =}\StringTok{ }\KeywordTok{file.path}\NormalTok{(}\StringTok{"Example_Data/bmi_age.txt"}\NormalTok{)}
\NormalTok{data_bmi =}\StringTok{ }\KeywordTok{read.table}\NormalTok{(}\DataTypeTok{file =}\NormalTok{ file_name, }\DataTypeTok{header =} \OtherTok{TRUE}\NormalTok{, }\DataTypeTok{stringsAsFactors =} \OtherTok{FALSE}\NormalTok{) }\CommentTok{# found elsewhere in the book. Downloaded from the repository}
\KeywordTok{colnames}\NormalTok{(data_bmi)}
\end{Highlighting}
\end{Shaded}

\begin{verbatim}
## [1] "PID" "BMI" "SEX" "AGE"
\end{verbatim}

\begin{enumerate}
\def\labelenumi{\alph{enumi}.}
\setcounter{enumi}{1}
\tightlist
\item
  Subset the AGE column using data\_bmi{[}, ``AGE''{]}.
\end{enumerate}

\begin{Shaded}
\begin{Highlighting}[]
\NormalTok{file_name =}\StringTok{ }\KeywordTok{file.path}\NormalTok{(}\StringTok{"Example_Data/bmi_age.txt"}\NormalTok{)}
\NormalTok{data_bmi =}\StringTok{ }\KeywordTok{read.table}\NormalTok{(}\DataTypeTok{file =}\NormalTok{ file_name, }\DataTypeTok{header =} \OtherTok{TRUE}\NormalTok{, }\DataTypeTok{stringsAsFactors =} \OtherTok{FALSE}\NormalTok{)}
\NormalTok{data_bmi[, }\StringTok{"AGE"}\NormalTok{]}
\end{Highlighting}
\end{Shaded}

\begin{verbatim}
##  [1] 45 57 66 49 33 40 65 59 65 42
\end{verbatim}

\begin{enumerate}
\def\labelenumi{\alph{enumi}.}
\setcounter{enumi}{2}
\tightlist
\item
  Subset the AGE column using data\_bmi\$AGE.
\end{enumerate}

\begin{Shaded}
\begin{Highlighting}[]
\NormalTok{file_name =}\StringTok{ }\KeywordTok{file.path}\NormalTok{(}\StringTok{"Example_Data/bmi_age.txt"}\NormalTok{)}
\NormalTok{data_bmi =}\StringTok{ }\KeywordTok{read.table}\NormalTok{(}\DataTypeTok{file =}\NormalTok{ file_name, }\DataTypeTok{header =} \OtherTok{TRUE}\NormalTok{, }\DataTypeTok{stringsAsFactors =} \OtherTok{FALSE}\NormalTok{)}
\NormalTok{data_bmi}\OperatorTok{$}\NormalTok{AGE}
\end{Highlighting}
\end{Shaded}

\begin{verbatim}
##  [1] 45 57 66 49 33 40 65 59 65 42
\end{verbatim}

\begin{enumerate}
\def\labelenumi{\alph{enumi}.}
\setcounter{enumi}{3}
\tightlist
\item
  Subset the AGE and BMI columns using brackets and the c() function.
\end{enumerate}

\begin{Shaded}
\begin{Highlighting}[]
\NormalTok{file_name =}\StringTok{ }\KeywordTok{file.path}\NormalTok{(}\StringTok{"Example_Data/bmi_age.txt"}\NormalTok{)}
\NormalTok{data_bmi =}\StringTok{ }\KeywordTok{read.table}\NormalTok{(}\DataTypeTok{file =}\NormalTok{ file_name, }\DataTypeTok{header =} \OtherTok{TRUE}\NormalTok{, }\DataTypeTok{stringsAsFactors =} \OtherTok{FALSE}\NormalTok{)}
\NormalTok{data_bmi[}\KeywordTok{c}\NormalTok{(}\StringTok{"AGE"}\NormalTok{, }\StringTok{"BMI"}\NormalTok{)]}
\end{Highlighting}
\end{Shaded}

\begin{verbatim}
##    AGE BMI
## 1   45  22
## 2   57  27
## 3   66  31
## 4   49  24
## 5   33  23
## 6   40  18
## 7   65  21
## 8   59  26
## 9   65  34
## 10  42  20
\end{verbatim}

\hypertarget{problem-4.-here-we-will-work-with-sequences-and-lengths}{%
\subsubsection{Problem 4. Here we will work with sequences and
lengths:}\label{problem-4.-here-we-will-work-with-sequences-and-lengths}}

\begin{enumerate}
\def\labelenumi{\alph{enumi}.}
\tightlist
\item
  Create a sequence from 1 to 4.5 by 0.24. Call this run\_num.
\end{enumerate}

\begin{Shaded}
\begin{Highlighting}[]
\NormalTok{run_num <-}\KeywordTok{seq}\NormalTok{(}\DataTypeTok{from =} \DecValTok{1}\NormalTok{, }\DataTypeTok{to =} \FloatTok{4.5}\NormalTok{, }\DataTypeTok{by =} \FloatTok{0.24}\NormalTok{)}
\end{Highlighting}
\end{Shaded}

\begin{enumerate}
\def\labelenumi{\alph{enumi}.}
\setcounter{enumi}{1}
\tightlist
\item
  What is the length of run\_num.
\end{enumerate}

\begin{Shaded}
\begin{Highlighting}[]
\KeywordTok{length}\NormalTok{(run_num)}
\end{Highlighting}
\end{Shaded}

\begin{verbatim}
## [1] 15
\end{verbatim}

\begin{enumerate}
\def\labelenumi{\alph{enumi}.}
\setcounter{enumi}{2}
\tightlist
\item
  Extract the fifth element of run\_num using brackets.
\end{enumerate}

\begin{Shaded}
\begin{Highlighting}[]
\NormalTok{run_num[}\DecValTok{5}\NormalTok{]}
\end{Highlighting}
\end{Shaded}

\begin{verbatim}
## [1] 1.96
\end{verbatim}

\hypertarget{problem-5.-lets-create-a-tibble-called-df}{%
\subsubsection{Problem 5. Let's create a tibble called
df:}\label{problem-5.-lets-create-a-tibble-called-df}}

\begin{enumerate}
\def\labelenumi{\alph{enumi}.}
\tightlist
\item
  Extract the column x using the \$.
\end{enumerate}

\begin{Shaded}
\begin{Highlighting}[]
\NormalTok{df =}\StringTok{ }\NormalTok{dplyr}\OperatorTok{::}\KeywordTok{data_frame}\NormalTok{(}\DataTypeTok{x =} \KeywordTok{rnorm}\NormalTok{(}\DecValTok{10}\NormalTok{), }\DataTypeTok{y =} \KeywordTok{rnorm}\NormalTok{(}\DecValTok{10}\NormalTok{), }\DataTypeTok{z =} \KeywordTok{rnorm}\NormalTok{(}\DecValTok{10}\NormalTok{)) }
\end{Highlighting}
\end{Shaded}

\begin{verbatim}
## Warning: `data_frame()` is deprecated, use `tibble()`.
## This warning is displayed once per session.
\end{verbatim}

\begin{Shaded}
\begin{Highlighting}[]
\NormalTok{df}\OperatorTok{$}\NormalTok{x}
\end{Highlighting}
\end{Shaded}

\begin{verbatim}
##  [1] -0.81850582  0.94564409  2.04209952  0.98579815 -0.65734723
##  [6]  0.89545332 -0.68871295 -0.47146923  0.04913984  0.78580341
\end{verbatim}

\begin{enumerate}
\def\labelenumi{\alph{enumi}.}
\setcounter{enumi}{1}
\tightlist
\item
  Extract the column x using the {[},{]} notation.
\end{enumerate}

\begin{Shaded}
\begin{Highlighting}[]
\NormalTok{df =}\StringTok{ }\NormalTok{dplyr}\OperatorTok{::}\KeywordTok{data_frame}\NormalTok{(}\DataTypeTok{x =} \KeywordTok{rnorm}\NormalTok{(}\DecValTok{10}\NormalTok{), }\DataTypeTok{y =} \KeywordTok{rnorm}\NormalTok{(}\DecValTok{10}\NormalTok{), }\DataTypeTok{z =} \KeywordTok{rnorm}\NormalTok{(}\DecValTok{10}\NormalTok{)) }
\NormalTok{df[,}\StringTok{"x"}\NormalTok{]}
\end{Highlighting}
\end{Shaded}

\begin{verbatim}
## # A tibble: 10 x 1
##         x
##     <dbl>
##  1 -1.91 
##  2  1.50 
##  3 -2.29 
##  4  0.663
##  5  0.745
##  6 -0.913
##  7  0.455
##  8  1.17 
##  9  0.310
## 10 -1.57
\end{verbatim}

\begin{enumerate}
\def\labelenumi{\alph{enumi}.}
\setcounter{enumi}{2}
\tightlist
\item
  Extract columns x and z.
\end{enumerate}

\begin{Shaded}
\begin{Highlighting}[]
\NormalTok{df =}\StringTok{ }\NormalTok{dplyr}\OperatorTok{::}\KeywordTok{data_frame}\NormalTok{(}\DataTypeTok{x =} \KeywordTok{rnorm}\NormalTok{(}\DecValTok{10}\NormalTok{), }\DataTypeTok{y =} \KeywordTok{rnorm}\NormalTok{(}\DecValTok{10}\NormalTok{), }\DataTypeTok{z =} \KeywordTok{rnorm}\NormalTok{(}\DecValTok{10}\NormalTok{)) }
\NormalTok{df[}\KeywordTok{c}\NormalTok{(}\StringTok{"x"}\NormalTok{, }\StringTok{"y"}\NormalTok{)]}
\end{Highlighting}
\end{Shaded}

\begin{verbatim}
## # A tibble: 10 x 2
##         x      y
##     <dbl>  <dbl>
##  1 -1.41   0.986
##  2  1.31  -0.414
##  3  1.19   0.400
##  4  0.752 -0.313
##  5 -1.04  -0.500
##  6 -1.18   0.930
##  7 -0.540  0.338
##  8 -0.632 -1.84 
##  9  1.51   0.496
## 10  1.42   0.755
\end{verbatim}

\begin{enumerate}
\def\labelenumi{\alph{enumi}.}
\setcounter{enumi}{3}
\tightlist
\item
  Extract the third and fifth rows of df and columns z and y.
\end{enumerate}

\begin{Shaded}
\begin{Highlighting}[]
\NormalTok{df =}\StringTok{ }\NormalTok{dplyr}\OperatorTok{::}\KeywordTok{data_frame}\NormalTok{(}\DataTypeTok{x =} \KeywordTok{rnorm}\NormalTok{(}\DecValTok{10}\NormalTok{), }\DataTypeTok{y =} \KeywordTok{rnorm}\NormalTok{(}\DecValTok{10}\NormalTok{), }\DataTypeTok{z =} \KeywordTok{rnorm}\NormalTok{(}\DecValTok{10}\NormalTok{)) }
\NormalTok{df[}\KeywordTok{c}\NormalTok{(}\DecValTok{3}\NormalTok{,}\DecValTok{5}\NormalTok{),]}
\end{Highlighting}
\end{Shaded}

\begin{verbatim}
## # A tibble: 2 x 3
##        x      y      z
##    <dbl>  <dbl>  <dbl>
## 1 -0.200  0.250  0.810
## 2 -0.523 -0.194 -2.54
\end{verbatim}

\begin{Shaded}
\begin{Highlighting}[]
\NormalTok{df[}\KeywordTok{c}\NormalTok{(}\StringTok{"x"}\NormalTok{,}\StringTok{"y"}\NormalTok{)]}
\end{Highlighting}
\end{Shaded}

\begin{verbatim}
## # A tibble: 10 x 2
##          x       y
##      <dbl>   <dbl>
##  1  0.117  -0.0960
##  2  0.264   2.46  
##  3 -0.200   0.250 
##  4  1.41   -0.831 
##  5 -0.523  -0.194 
##  6  2.24    2.70  
##  7 -0.0119  0.0583
##  8  1.16    1.06  
##  9  0.195   0.244 
## 10  0.854  -0.348
\end{verbatim}

\hypertarget{problem-6}{%
\subsubsection{Problem 6}\label{problem-6}}

\begin{enumerate}
\def\labelenumi{\alph{enumi}.}
\tightlist
\item
  Get the mean of the column x using the \$ operator.
\end{enumerate}

\begin{Shaded}
\begin{Highlighting}[]
\NormalTok{df =}\StringTok{ }\NormalTok{dplyr}\OperatorTok{::}\KeywordTok{data_frame}\NormalTok{(}\DataTypeTok{x =} \KeywordTok{rnorm}\NormalTok{(}\DecValTok{10}\NormalTok{), }\DataTypeTok{y =} \KeywordTok{rnorm}\NormalTok{(}\DecValTok{10}\NormalTok{), }\DataTypeTok{z =} \KeywordTok{rnorm}\NormalTok{(}\DecValTok{10}\NormalTok{))}
\KeywordTok{mean}\NormalTok{(df}\OperatorTok{$}\NormalTok{x)}
\end{Highlighting}
\end{Shaded}

\begin{verbatim}
## [1] -0.2418342
\end{verbatim}

\begin{enumerate}
\def\labelenumi{\alph{enumi}.}
\setcounter{enumi}{1}
\tightlist
\item
  Pipe (\%\textgreater{}\%) the column x into the mean function.
\end{enumerate}

\begin{Shaded}
\begin{Highlighting}[]
\KeywordTok{library}\NormalTok{(dplyr)}
\end{Highlighting}
\end{Shaded}

\begin{verbatim}
## 
## Attaching package: 'dplyr'
\end{verbatim}

\begin{verbatim}
## The following objects are masked from 'package:stats':
## 
##     filter, lag
\end{verbatim}

\begin{verbatim}
## The following objects are masked from 'package:base':
## 
##     intersect, setdiff, setequal, union
\end{verbatim}

\begin{Shaded}
\begin{Highlighting}[]
\NormalTok{df}\OperatorTok{$}\NormalTok{x }\OperatorTok\StringTok{ }\NormalTok{mean}
\end{Highlighting}
\end{Shaded}

\begin{verbatim}
## [1] -0.2418342
\end{verbatim}

\begin{enumerate}
\def\labelenumi{\alph{enumi}.}
\setcounter{enumi}{2}
\tightlist
\item
  Look at the summarize function to look ahead to future chapters and
  try to do this in the dplyr framework.
\end{enumerate}

\begin{Shaded}
\begin{Highlighting}[]
\KeywordTok{print}\NormalTok{(}\KeywordTok{paste}\NormalTok{(}\StringTok{"K"}\NormalTok{))}
\end{Highlighting}
\end{Shaded}

\begin{verbatim}
## [1] "K"
\end{verbatim}

\hypertarget{problem-7.}{%
\subsubsection{Problem 7.}\label{problem-7.}}

Consider the data set data\_bmi using the BMI data, but read in using
readr:

\begin{Shaded}
\begin{Highlighting}[]
\NormalTok{file_name =}\StringTok{ "Example_Data/bmi_age.txt"}
\NormalTok{data_bmi =}\StringTok{ }\NormalTok{readr}\OperatorTok{::}\KeywordTok{read_table2}\NormalTok{(}\DataTypeTok{file =}\NormalTok{ file_name) }
\end{Highlighting}
\end{Shaded}

\begin{verbatim}
## Warning: Missing column names filled in: 'X5' [5]
\end{verbatim}

\begin{verbatim}
## Parsed with column specification:
## cols(
##   PID = col_double(),
##   BMI = col_double(),
##   SEX = col_double(),
##   AGE = col_double(),
##   X5 = col_logical()
## )
\end{verbatim}

\begin{verbatim}
## Warning: 3 parsing failures.
## row col  expected    actual                       file
##   2  -- 5 columns 4 columns 'Example_Data/bmi_age.txt'
##   3  -- 5 columns 4 columns 'Example_Data/bmi_age.txt'
##  10  -- 5 columns 4 columns 'Example_Data/bmi_age.txt'
\end{verbatim}

\begin{enumerate}
\def\labelenumi{\alph{enumi}.}
\tightlist
\item
  What is the class of data\_bmi?
\end{enumerate}

\begin{Shaded}
\begin{Highlighting}[]
\NormalTok{file_name =}\StringTok{ "Example_Data/bmi_age.txt"}
\NormalTok{data_bmi =}\StringTok{ }\NormalTok{readr}\OperatorTok{::}\KeywordTok{read_table2}\NormalTok{(}\DataTypeTok{file =}\NormalTok{ file_name)}
\end{Highlighting}
\end{Shaded}

\begin{verbatim}
## Warning: Missing column names filled in: 'X5' [5]
\end{verbatim}

\begin{verbatim}
## Parsed with column specification:
## cols(
##   PID = col_double(),
##   BMI = col_double(),
##   SEX = col_double(),
##   AGE = col_double(),
##   X5 = col_logical()
## )
\end{verbatim}

\begin{verbatim}
## Warning: 3 parsing failures.
## row col  expected    actual                       file
##   2  -- 5 columns 4 columns 'Example_Data/bmi_age.txt'
##   3  -- 5 columns 4 columns 'Example_Data/bmi_age.txt'
##  10  -- 5 columns 4 columns 'Example_Data/bmi_age.txt'
\end{verbatim}

\begin{Shaded}
\begin{Highlighting}[]
\KeywordTok{class}\NormalTok{(data_bmi)}
\end{Highlighting}
\end{Shaded}

\begin{verbatim}
## [1] "spec_tbl_df" "tbl_df"      "tbl"         "data.frame"
\end{verbatim}

\begin{enumerate}
\def\labelenumi{\alph{enumi}.}
\setcounter{enumi}{1}
\tightlist
\item
  What is the class of data\_bmi{[}, ``AGE''{]}?
\end{enumerate}

\begin{Shaded}
\begin{Highlighting}[]
\NormalTok{file_name =}\StringTok{ "Example_Data/bmi_age.txt"}
\NormalTok{data_bmi =}\StringTok{ }\NormalTok{readr}\OperatorTok{::}\KeywordTok{read_table2}\NormalTok{(}\DataTypeTok{file =}\NormalTok{ file_name)}
\end{Highlighting}
\end{Shaded}

\begin{verbatim}
## Warning: Missing column names filled in: 'X5' [5]
\end{verbatim}

\begin{verbatim}
## Parsed with column specification:
## cols(
##   PID = col_double(),
##   BMI = col_double(),
##   SEX = col_double(),
##   AGE = col_double(),
##   X5 = col_logical()
## )
\end{verbatim}

\begin{verbatim}
## Warning: 3 parsing failures.
## row col  expected    actual                       file
##   2  -- 5 columns 4 columns 'Example_Data/bmi_age.txt'
##   3  -- 5 columns 4 columns 'Example_Data/bmi_age.txt'
##  10  -- 5 columns 4 columns 'Example_Data/bmi_age.txt'
\end{verbatim}

\begin{Shaded}
\begin{Highlighting}[]
\KeywordTok{class}\NormalTok{(data_bmi[, }\StringTok{"AGE"}\NormalTok{])}
\end{Highlighting}
\end{Shaded}

\begin{verbatim}
## [1] "tbl_df"     "tbl"        "data.frame"
\end{verbatim}

\begin{enumerate}
\def\labelenumi{\alph{enumi}.}
\setcounter{enumi}{2}
\tightlist
\item
  What is the class of data\_bmi\$AGE?
\end{enumerate}

\begin{Shaded}
\begin{Highlighting}[]
\NormalTok{file_name =}\StringTok{ "Example_Data/bmi_age.txt"}
\NormalTok{data_bmi =}\StringTok{ }\NormalTok{readr}\OperatorTok{::}\KeywordTok{read_table2}\NormalTok{(}\DataTypeTok{file =}\NormalTok{ file_name)}
\end{Highlighting}
\end{Shaded}

\begin{verbatim}
## Warning: Missing column names filled in: 'X5' [5]
\end{verbatim}

\begin{verbatim}
## Parsed with column specification:
## cols(
##   PID = col_double(),
##   BMI = col_double(),
##   SEX = col_double(),
##   AGE = col_double(),
##   X5 = col_logical()
## )
\end{verbatim}

\begin{verbatim}
## Warning: 3 parsing failures.
## row col  expected    actual                       file
##   2  -- 5 columns 4 columns 'Example_Data/bmi_age.txt'
##   3  -- 5 columns 4 columns 'Example_Data/bmi_age.txt'
##  10  -- 5 columns 4 columns 'Example_Data/bmi_age.txt'
\end{verbatim}

\begin{Shaded}
\begin{Highlighting}[]
\KeywordTok{class}\NormalTok{(data_bmi}\OperatorTok{$}\NormalTok{AGE)}
\end{Highlighting}
\end{Shaded}

\begin{verbatim}
## [1] "numeric"
\end{verbatim}

\begin{enumerate}
\def\labelenumi{\alph{enumi}.}
\setcounter{enumi}{3}
\tightlist
\item
  What is the class of data\_bmi{[}, ``AGE'', drop = TRUE{]}?
\end{enumerate}

\begin{Shaded}
\begin{Highlighting}[]
\NormalTok{file_name =}\StringTok{ "Example_Data/bmi_age.txt"}
\NormalTok{data_bmi =}\StringTok{ }\NormalTok{readr}\OperatorTok{::}\KeywordTok{read_table2}\NormalTok{(}\DataTypeTok{file =}\NormalTok{ file_name)}
\end{Highlighting}
\end{Shaded}

\begin{verbatim}
## Warning: Missing column names filled in: 'X5' [5]
\end{verbatim}

\begin{verbatim}
## Parsed with column specification:
## cols(
##   PID = col_double(),
##   BMI = col_double(),
##   SEX = col_double(),
##   AGE = col_double(),
##   X5 = col_logical()
## )
\end{verbatim}

\begin{verbatim}
## Warning: 3 parsing failures.
## row col  expected    actual                       file
##   2  -- 5 columns 4 columns 'Example_Data/bmi_age.txt'
##   3  -- 5 columns 4 columns 'Example_Data/bmi_age.txt'
##  10  -- 5 columns 4 columns 'Example_Data/bmi_age.txt'
\end{verbatim}

\begin{Shaded}
\begin{Highlighting}[]
\KeywordTok{class}\NormalTok{(data_bmi[, }\StringTok{"AGE"}\NormalTok{, }\DataTypeTok{drop =} \OtherTok{TRUE}\NormalTok{])}
\end{Highlighting}
\end{Shaded}

\begin{verbatim}
## [1] "numeric"
\end{verbatim}

\hypertarget{problem-8.-consider-the-data-set-data_bmi-using-the-bmi-data-but-read-in-using-readr}{%
\subsubsection{Problem 8. Consider the data set data\_bmi using the BMI
data, but read in using
readr:}\label{problem-8.-consider-the-data-set-data_bmi-using-the-bmi-data-but-read-in-using-readr}}

file\_name = ``bmi\_age.txt'' data\_bmi = readr::read\_table2(file =
file\_name) Parsed with column specification:

cols( PID = col\_double(), BMI = col\_double(), SEX = col\_double(), AGE
= col\_double() )

\begin{enumerate}
\def\labelenumi{\alph{enumi}.}
\tightlist
\item
  What is the mean of the AGE column?
\end{enumerate}

\begin{Shaded}
\begin{Highlighting}[]
\NormalTok{file_name =}\StringTok{ "Example_Data/bmi_age.txt"}
\NormalTok{data_bmi =}\StringTok{ }\NormalTok{readr}\OperatorTok{::}\KeywordTok{read_table2}\NormalTok{(}\DataTypeTok{file =}\NormalTok{ file_name)}
\end{Highlighting}
\end{Shaded}

\begin{verbatim}
## Warning: Missing column names filled in: 'X5' [5]
\end{verbatim}

\begin{verbatim}
## Parsed with column specification:
## cols(
##   PID = col_double(),
##   BMI = col_double(),
##   SEX = col_double(),
##   AGE = col_double(),
##   X5 = col_logical()
## )
\end{verbatim}

\begin{verbatim}
## Warning: 3 parsing failures.
## row col  expected    actual                       file
##   2  -- 5 columns 4 columns 'Example_Data/bmi_age.txt'
##   3  -- 5 columns 4 columns 'Example_Data/bmi_age.txt'
##  10  -- 5 columns 4 columns 'Example_Data/bmi_age.txt'
\end{verbatim}

\begin{Shaded}
\begin{Highlighting}[]
\KeywordTok{mean}\NormalTok{(data_bmi}\OperatorTok{$}\NormalTok{AGE)}
\end{Highlighting}
\end{Shaded}

\begin{verbatim}
## [1] 52.1
\end{verbatim}

\begin{enumerate}
\def\labelenumi{\alph{enumi}.}
\setcounter{enumi}{1}
\tightlist
\item
  Set the 3rd element of data\_bmi\$AGE to be 42?
\end{enumerate}

\begin{Shaded}
\begin{Highlighting}[]
\NormalTok{file_name =}\StringTok{ "Example_Data/bmi_age.txt"}
\NormalTok{data_bmi =}\StringTok{ }\NormalTok{readr}\OperatorTok{::}\KeywordTok{read_table2}\NormalTok{(}\DataTypeTok{file =}\NormalTok{ file_name)}
\end{Highlighting}
\end{Shaded}

\begin{verbatim}
## Warning: Missing column names filled in: 'X5' [5]
\end{verbatim}

\begin{verbatim}
## Parsed with column specification:
## cols(
##   PID = col_double(),
##   BMI = col_double(),
##   SEX = col_double(),
##   AGE = col_double(),
##   X5 = col_logical()
## )
\end{verbatim}

\begin{verbatim}
## Warning: 3 parsing failures.
## row col  expected    actual                       file
##   2  -- 5 columns 4 columns 'Example_Data/bmi_age.txt'
##   3  -- 5 columns 4 columns 'Example_Data/bmi_age.txt'
##  10  -- 5 columns 4 columns 'Example_Data/bmi_age.txt'
\end{verbatim}

\begin{Shaded}
\begin{Highlighting}[]
\NormalTok{data_bmi}\OperatorTok{$}\NormalTok{AGE[}\DecValTok{3}\NormalTok{] =}\StringTok{ }\DecValTok{42}
\end{Highlighting}
\end{Shaded}

\begin{enumerate}
\def\labelenumi{\alph{enumi}.}
\setcounter{enumi}{2}
\tightlist
\item
  What is the mean of the AGE column now?
\end{enumerate}

\begin{Shaded}
\begin{Highlighting}[]
\NormalTok{file_name =}\StringTok{ "Example_Data/bmi_age.txt"}
\NormalTok{data_bmi =}\StringTok{ }\NormalTok{readr}\OperatorTok{::}\KeywordTok{read_table2}\NormalTok{(}\DataTypeTok{file =}\NormalTok{ file_name)}
\end{Highlighting}
\end{Shaded}

\begin{verbatim}
## Warning: Missing column names filled in: 'X5' [5]
\end{verbatim}

\begin{verbatim}
## Parsed with column specification:
## cols(
##   PID = col_double(),
##   BMI = col_double(),
##   SEX = col_double(),
##   AGE = col_double(),
##   X5 = col_logical()
## )
\end{verbatim}

\begin{verbatim}
## Warning: 3 parsing failures.
## row col  expected    actual                       file
##   2  -- 5 columns 4 columns 'Example_Data/bmi_age.txt'
##   3  -- 5 columns 4 columns 'Example_Data/bmi_age.txt'
##  10  -- 5 columns 4 columns 'Example_Data/bmi_age.txt'
\end{verbatim}

\begin{Shaded}
\begin{Highlighting}[]
\KeywordTok{mean}\NormalTok{(data_bmi}\OperatorTok{$}\NormalTok{AGE)}
\end{Highlighting}
\end{Shaded}

\begin{verbatim}
## [1] 52.1
\end{verbatim}

\hypertarget{problem-9.-use-data_bmi-from-the-above-problem}{%
\subsubsection{Problem 9. Use data\_bmi from the above
problem:}\label{problem-9.-use-data_bmi-from-the-above-problem}}

\begin{enumerate}
\def\labelenumi{\alph{enumi}.}
\tightlist
\item
  Remove the X5 column using data\_bmi\$X5 = NULL
\end{enumerate}

\begin{Shaded}
\begin{Highlighting}[]
\NormalTok{file_name =}\StringTok{ "Example_Data/bmi_age.txt"}
\NormalTok{data_bmi =}\StringTok{ }\NormalTok{readr}\OperatorTok{::}\KeywordTok{read_table2}\NormalTok{(}\DataTypeTok{file =}\NormalTok{ file_name)}
\end{Highlighting}
\end{Shaded}

\begin{verbatim}
## Warning: Missing column names filled in: 'X5' [5]
\end{verbatim}

\begin{verbatim}
## Parsed with column specification:
## cols(
##   PID = col_double(),
##   BMI = col_double(),
##   SEX = col_double(),
##   AGE = col_double(),
##   X5 = col_logical()
## )
\end{verbatim}

\begin{verbatim}
## Warning: 3 parsing failures.
## row col  expected    actual                       file
##   2  -- 5 columns 4 columns 'Example_Data/bmi_age.txt'
##   3  -- 5 columns 4 columns 'Example_Data/bmi_age.txt'
##  10  -- 5 columns 4 columns 'Example_Data/bmi_age.txt'
\end{verbatim}

\begin{Shaded}
\begin{Highlighting}[]
\NormalTok{data_bmi}\OperatorTok{$}\NormalTok{X5 =}\StringTok{ }\OtherTok{NULL}
\end{Highlighting}
\end{Shaded}

\begin{enumerate}
\def\labelenumi{\alph{enumi}.}
\setcounter{enumi}{1}
\tightlist
\item
  Create mat, which is data\_bmi as a matrix, using as.matrix.
\end{enumerate}

\begin{Shaded}
\begin{Highlighting}[]
\NormalTok{mat <-}\StringTok{ }\KeywordTok{as.matrix}\NormalTok{(data_bmi)}
\end{Highlighting}
\end{Shaded}

\begin{enumerate}
\def\labelenumi{\alph{enumi}.}
\setcounter{enumi}{2}
\tightlist
\item
  Try to extract AGE from mat using the \$. What happened?
\end{enumerate}

\begin{Shaded}
\begin{Highlighting}[]
\CommentTok{#It causes a fatal error}
\CommentTok{#mat$AGE}
\end{Highlighting}
\end{Shaded}

\begin{enumerate}
\def\labelenumi{\alph{enumi}.}
\setcounter{enumi}{3}
\tightlist
\item
  Extract AGE from mat using the {[},{]} notation.
\end{enumerate}

\begin{Shaded}
\begin{Highlighting}[]
\NormalTok{mat[,}\StringTok{"AGE"}\NormalTok{]}
\end{Highlighting}
\end{Shaded}

\begin{verbatim}
##  [1] 45 57 66 49 33 40 65 59 65 42
\end{verbatim}

\hypertarget{chapter-3.-probability-random-variables-distributions}{%
\section{Chapter 3. Probability, random variables,
distributions}\label{chapter-3.-probability-random-variables-distributions}}

\hypertarget{experiments}{%
\subsection{3.1 Experiments}\label{experiments}}

``\ldots{}.An experiment is the process of data collection for a target
population according to a specific sampling protocol that includes rules
for what, when, where, and how to collect data on experimental units
(e.g.~individuals) from the target population\ldots{}.''

Consider the outcome of an experiment such as:

\begin{itemize}
\tightlist
\item
  a collection of measurements from a sampled population
\item
  measurements from a laboratory experiment
\item
  the result of a clinical trial
\item
  the results of a simulated (computer) experiment
\item
  values from hospital records sampled retrospectively
\end{itemize}

\hypertarget{notation}{%
\subsection{3.1.1 Notation}\label{notation}}

To develop these concepts rigorously, we will introduce some notation
for the experiment setup. * The sample space Ω is the collection of
possible outcomes of an experiment Example: a six-sided die roll = \{1;
2; 3; 4; 5; 6\}, a coin flip = \{heads; tails\} * An event, say E, is a
subset of Ω Example: die roll is even E = \{2; 4; 6\} * An elementary or
simple event is a particular result of an experiment Example: die roll
is a four, ω = 4 * ∅ is called the null event or the empty set

``\ldots{}.Biostatis- tics is concerned with extracting useful,
actionable information from complex outcomes that can result from even
the simplest experiments\ldots{}.''

\hypertarget{interpretation-of-set-operations}{%
\subsection{3.1.2 Interpretation of set
operations}\label{interpretation-of-set-operations}}

``\ldots{}.Data analysis, manipulation, and tabulation are intrinsically
related to data sets and logic operators. Indeed, we often try to
understand the structure of the data and that requires understanding how
to subset the data and how to quantify the various relationships between
subsets of the data. To better understand this, we need to introduce the
theoretical set operations and their interpretations\ldots{}.''

\begin{itemize}
\tightlist
\item
  ω ∈ E implies that E occurs when ω occurs
\item
  ω ∈/E implies that E does not occur when ω occurs
\item
  E ⊂ F implies that the occurrence of E implies the occurrence of F (E
  is a subset of F )
\item
  E ∩ F implies the event that both E and F occur (intersection)
\item
  E ∪ F implies the event that at least one of E or F occur (union)
\item
  E ∩ F = ∅ means that E and F are mutually exclusive/disjoint, or
  cannot both occur
\item
  E c or Ē is the event that E does not occur, also referred to as the
  com- plement
\item
  E ~F = E ∩ F c is the event that E occurs and F does not occur
\end{itemize}

\hypertarget{set-theory-facts}{%
\subsection{3.1.3 Set theory facts}\label{set-theory-facts}}

DeMorgan's laws.

\begin{itemize}
\tightlist
\item
  (A ∪ B)\textsuperscript{c} = Ac ∩ B\textsuperscript{c}
\item
  (A ∩ B)\textsuperscript{c} = Ac ∪ B\textsuperscript{c}
\item
  (A\textsuperscript{c})\textsuperscript{c} = A
\item
  (A ∪ B) ~C = (A ~C) ∪ (B ~C) where ~means ``set minus''
\end{itemize}

``\ldots{}Proving the equality of two sets, A = B, is done by showing
that every element of A is in B (i.e., A ⊂ B) and every element in B is
in A (i.e., B ⊂ A)\ldots{}.''

"\ldots{}.In general, set operations are based on logical operators: AND
(in R \&), OR (in R \textbar{} ), NOT (in R !). These logical operators
can be combined and can produce complex combinations that are extremely
useful in practice. For example, hav- ing a data set one may want to
focus on a subset that contains only African American men from age 65 to
75 who are non-smokers. This phrase can im- mediately be translated into
logical operators and the data can be extracted using R programming.
This is a routine application of set theory that will be- come
indispensable in practice (below we provide an example of exactly how to
proceed).

``\ldots{}.We start by reading a small data set. First, let us tell R
where the file is: The data are loaded using the read.table function We
specify arguments/options to read.table that the file contains a header
row (header = TRUE) and we do not want to convert strings/words to a
categorical variable/factor when the data are read in (stringsAsFactors
= FALSE)\ldots{}.We can simply write the object name out to show the
data (here the entire data set is displayed because it is
small)\ldots{}.''

\begin{Shaded}
\begin{Highlighting}[]
\NormalTok{file_name =}\StringTok{ }\KeywordTok{file.path}\NormalTok{(}\StringTok{"Example_Data/bmi_age.txt"}\NormalTok{)}
\NormalTok{data_bmi =}\StringTok{ }\KeywordTok{read.table}\NormalTok{(}\DataTypeTok{file =}\NormalTok{ file_name, }\DataTypeTok{header =} \OtherTok{TRUE}\NormalTok{, }\DataTypeTok{stringsAsFactors =} \OtherTok{FALSE}\NormalTok{)}
\NormalTok{data_bmi }\CommentTok{#show the data}
\end{Highlighting}
\end{Shaded}

\begin{verbatim}
##    PID BMI SEX AGE
## 1    1  22   1  45
## 2    2  27   0  57
## 3    3  31   1  66
## 4    4  24   1  49
## 5    5  23   0  33
## 6    6  18   0  40
## 7    7  21   0  65
## 8    8  26   1  59
## 9    9  34   1  65
## 10  10  20   0  42
\end{verbatim}

\begin{Shaded}
\begin{Highlighting}[]
\KeywordTok{head}\NormalTok{(data_bmi) }\CommentTok{#show only a portion of the data}
\end{Highlighting}
\end{Shaded}

\begin{verbatim}
##   PID BMI SEX AGE
## 1   1  22   1  45
## 2   2  27   0  57
## 3   3  31   1  66
## 4   4  24   1  49
## 5   5  23   0  33
## 6   6  18   0  40
\end{verbatim}

\begin{Shaded}
\begin{Highlighting}[]
\KeywordTok{dim}\NormalTok{(data_bmi) }\CommentTok{#display the dimension of the data}
\end{Highlighting}
\end{Shaded}

\begin{verbatim}
## [1] 10  4
\end{verbatim}

\begin{Shaded}
\begin{Highlighting}[]
\KeywordTok{colnames}\NormalTok{(data_bmi) }\CommentTok{#the name of the variables (column names):}
\end{Highlighting}
\end{Shaded}

\begin{verbatim}
## [1] "PID" "BMI" "SEX" "AGE"
\end{verbatim}

\begin{Shaded}
\begin{Highlighting}[]
\KeywordTok{mean}\NormalTok{(data_bmi}\OperatorTok{$}\NormalTok{BMI)  }
\end{Highlighting}
\end{Shaded}

\begin{verbatim}
## [1] 24.6
\end{verbatim}

\begin{Shaded}
\begin{Highlighting}[]
\KeywordTok{sd}\NormalTok{(data_bmi}\OperatorTok{$}\NormalTok{BMI) }\CommentTok{#standard deviation}
\end{Highlighting}
\end{Shaded}

\begin{verbatim}
## [1] 4.993329
\end{verbatim}

``\ldots{}.For simplicity, assign all the columns of the dataset to
their own separate variables using the attach function. We do not
recommend this approach in general, especially when multiple datasets
are being used\ldots{}.''

\begin{Shaded}
\begin{Highlighting}[]
\KeywordTok{attach}\NormalTok{(data_bmi)}
\end{Highlighting}
\end{Shaded}

A basic plot of age vs.~BMI:

\begin{Shaded}
\begin{Highlighting}[]
\KeywordTok{plot}\NormalTok{(AGE, BMI,}\DataTypeTok{type=}\StringTok{"p"}\NormalTok{,}\DataTypeTok{pch=}\DecValTok{20}\NormalTok{,}\DataTypeTok{cex=}\DecValTok{2}\NormalTok{,}\DataTypeTok{col=}\StringTok{"blue"}\NormalTok{) }
\KeywordTok{rect}\NormalTok{(}\DataTypeTok{xleft =} \DecValTok{45}\NormalTok{, }\DataTypeTok{xright =} \DecValTok{100}\NormalTok{, }\DataTypeTok{ybottom =} \DecValTok{0}\NormalTok{, }\DataTypeTok{ytop =} \DecValTok{26}\NormalTok{, }\DataTypeTok{col =}\NormalTok{ scales}\OperatorTok{::}\KeywordTok{alpha}\NormalTok{(}\StringTok{"purple"}\NormalTok{, }\FloatTok{0.5}\NormalTok{)) }
\KeywordTok{rect}\NormalTok{(}\DataTypeTok{xleft =} \DecValTok{45}\NormalTok{, }\DataTypeTok{xright =} \DecValTok{100}\NormalTok{, }\DataTypeTok{ybottom =} \DecValTok{26}\NormalTok{, }\DataTypeTok{ytop =} \DecValTok{100}\NormalTok{, }\DataTypeTok{col =}\NormalTok{ scales}\OperatorTok{::}\KeywordTok{alpha}\NormalTok{(}\StringTok{"red"}\NormalTok{, }\FloatTok{0.5}\NormalTok{)) }
\KeywordTok{rect}\NormalTok{(}\DataTypeTok{xleft =} \DecValTok{0}\NormalTok{, }\DataTypeTok{xright =} \DecValTok{45}\NormalTok{, }\DataTypeTok{ybottom =} \DecValTok{0}\NormalTok{, }\DataTypeTok{ytop =} \DecValTok{26}\NormalTok{, }\DataTypeTok{col =}\NormalTok{ scales}\OperatorTok{::}\KeywordTok{alpha}\NormalTok{(}\StringTok{"blue"}\NormalTok{, }\FloatTok{0.5}\NormalTok{)) }
\KeywordTok{points}\NormalTok{(AGE, BMI,}\DataTypeTok{type=}\StringTok{"p"}\NormalTok{,}\DataTypeTok{pch=}\DecValTok{20}\NormalTok{,}\DataTypeTok{cex=}\DecValTok{2}\NormalTok{,}\DataTypeTok{col=}\StringTok{"blue"}\NormalTok{) }
\KeywordTok{text}\NormalTok{(}\DataTypeTok{x =}\NormalTok{ AGE }\OperatorTok{+}\StringTok{ }\DecValTok{1}\NormalTok{, }\DataTypeTok{y =}\NormalTok{ BMI, }\DataTypeTok{labels =}\NormalTok{ PID, }\DataTypeTok{col =} \StringTok{"black"}\NormalTok{) }
\KeywordTok{abline}\NormalTok{(}\DataTypeTok{h =} \DecValTok{26}\NormalTok{, }\DataTypeTok{v =} \DecValTok{45}\NormalTok{, }\DataTypeTok{col =} \StringTok{"RED"}\NormalTok{)}
\end{Highlighting}
\end{Shaded}

\includegraphics{Notes_files/figure-latex/unnamed-chunk-51-1.pdf}

"\ldots{}.Here the call to abline is for drawing vertical and horizontal
lines so that we can count the numbers in each respective box. We now
make the connection between set theory and operations clearer in the
context of the data. In particular, we emphasize how set operations
translate into logic operators that can then be used for data extraction
and operations. Consider the following subsets of subjects:

A subjects with BMI\textless{} 26 and B subjects older than 45
(AGE\textgreater{} 45). Construct a logical indicator for which records
fall into A\ldots{}."

\begin{Shaded}
\begin{Highlighting}[]
\CommentTok{#Group A, #represented by the points in the purple/blue regions}
\NormalTok{index_BMI_less_}\DecValTok{26}\NormalTok{ =}\StringTok{ }\NormalTok{BMI }\OperatorTok{<}\StringTok{ }\DecValTok{26} 
\CommentTok{#Group B, #represented by the points in the purple/red regions }
\NormalTok{index_AGE_greater_}\DecValTok{45}\NormalTok{ =}\StringTok{ }\NormalTok{AGE }\OperatorTok{>}\StringTok{ }\DecValTok{45} 
\end{Highlighting}
\end{Shaded}

``\ldots{}.Display the IDs for A and B. Here PID is the unique patient
ID, though in many applications the PID can be more complex than just
integer numbers\ldots{}.''

\begin{Shaded}
\begin{Highlighting}[]
\NormalTok{PID[index_BMI_less_}\DecValTok{26}\NormalTok{]}
\end{Highlighting}
\end{Shaded}

\begin{verbatim}
## [1]  1  4  5  6  7 10
\end{verbatim}

\begin{Shaded}
\begin{Highlighting}[]
\NormalTok{PID[index_AGE_greater_}\DecValTok{45}\NormalTok{]}
\end{Highlighting}
\end{Shaded}

\begin{verbatim}
## [1] 2 3 4 7 8 9
\end{verbatim}

Let us calculate (A ∩ B)\textsuperscript{c} , the complement of the
intersection between A and B, which is shown by the non-purple regions
in Figure 3.1. These are subjects who do not/are (note the c that stands
for complement) (have a BMI less than 26) and (older than 45) and

\begin{Shaded}
\begin{Highlighting}[]
\NormalTok{index_A_int_B_compl =}\StringTok{ }\OperatorTok{!}\NormalTok{(index_BMI_less_}\DecValTok{26} \OperatorTok{&}\StringTok{ }\NormalTok{index_AGE_greater_}\DecValTok{45}\NormalTok{) }
\NormalTok{PID[index_A_int_B_compl]}
\end{Highlighting}
\end{Shaded}

\begin{verbatim}
## [1]  1  2  3  5  6  8  9 10
\end{verbatim}

The translation of (A∩B\textsuperscript{c} into R is
!(index\_BMI\_less\_26 \& index\_AGE\_greater\_45). Note that !
indicates is not, or complement, and \& indicates and, or intersection.
So, the resulting IDs are everybody, except the subject with IDs 4 and
7.

It would be instructive to conduct the same type of analysis for

A\textsuperscript{c} ∪ B\textsuperscript{c} = (A ∪ B)\textsuperscript{c}

\begin{Shaded}
\begin{Highlighting}[]
\NormalTok{index_A_int_B_compl =}\StringTok{ }\NormalTok{(index_BMI_less_}\DecValTok{26} \OperatorTok{&}\StringTok{ }\NormalTok{index_AGE_greater_}\DecValTok{45}\NormalTok{) }
\NormalTok{PID[index_A_int_B_compl]}
\end{Highlighting}
\end{Shaded}

\begin{verbatim}
## [1] 4 7
\end{verbatim}

(A ∪ B)\textsuperscript{c} ,

\begin{Shaded}
\begin{Highlighting}[]
\NormalTok{index_A_int_B_compl =}\StringTok{ }\NormalTok{(index_BMI_less_}\DecValTok{26} \OperatorTok{&}\StringTok{ }\NormalTok{index_AGE_greater_}\DecValTok{45}\NormalTok{) }
\NormalTok{PID[index_A_int_B_compl]}
\end{Highlighting}
\end{Shaded}

\begin{verbatim}
## [1] 4 7
\end{verbatim}

and A\textsuperscript{c}.

\begin{Shaded}
\begin{Highlighting}[]
\NormalTok{index_A_int_B_compl =}\StringTok{ }\NormalTok{(index_BMI_less_}\DecValTok{26}\NormalTok{)}
\NormalTok{PID[index_A_int_B_compl]}
\end{Highlighting}
\end{Shaded}

\begin{verbatim}
## [1]  1  4  5  6  7 10
\end{verbatim}

\hypertarget{an-intuitive-introduction-to-the-bootstrap}{%
\subsection{3.2 An intuitive introduction to the
bootstrap}\label{an-intuitive-introduction-to-the-bootstrap}}

"\ldots{}.A major problem in practice is that, even if we run two
identical experiments, data are never the same.

However, even though the two samples will be different, they will have
some things in common. Those things are the target of estimation, the
probability and time for conversion from being healthy to developing
lung cancer, and the original target population\ldots{}."

In practice we often have one study and we are interested in
understanding what would happen if multiple studies were conducted. The
reason for that is fundamental and it has to do with the
generalizability of the experiment. Indeed, a study would collect data
for a subsample of the population to make predictions about the rate and
expected time of transition to lung cancer in the overall population.
Bootstrap is a widely used statistical technique based on resampling the
data. The bootstrap is designed to create many potential studies that
share the same characteristics with the study that collected the data
and may represent the vari- ability of running these experiments without
actually running the experiments\ldots{}."

``\ldots{}.The nonparametric bootstrap is the procedure of resampling
with replacement from a dataset, where the number of observations in
each resampled dataset is equal to the number of observations in the
original dataset\ldots{}.''

So this is pretty much ``take multiple random samples, the game'' - Oh
and here's one way to do it:

\begin{Shaded}
\begin{Highlighting}[]
\KeywordTok{set.seed}\NormalTok{(}\DecValTok{4132697}\NormalTok{) }
\NormalTok{nboot=}\DecValTok{3} \CommentTok{#number of boostrap samples }
\NormalTok{nsubj=}\KeywordTok{nrow}\NormalTok{(data_bmi) }\CommentTok{#number of subjects }
\ControlFlowTok{for}\NormalTok{ (i }\ControlFlowTok{in} \DecValTok{1}\OperatorTok{:}\NormalTok{nboot) }\CommentTok{#start bootstrapping }
\NormalTok{  \{}\CommentTok{#begin the bootstrap for loop }
\NormalTok{    resampling =}\StringTok{ }\KeywordTok{sample}\NormalTok{(nsubj, }\DataTypeTok{replace=}\OtherTok{TRUE}\NormalTok{) }\CommentTok{#sample with replacement }
\NormalTok{    bootstrapped_data =}\StringTok{ }\NormalTok{data_bmi[resampling,] }\CommentTok{#resample the data set }
    \KeywordTok{print}\NormalTok{(bootstrapped_data) }\CommentTok{#print bootstrapped datasets }
    \KeywordTok{cat}\NormalTok{(}\StringTok{" }\CharTok{\textbackslash{}n}\StringTok{ }\CharTok{\textbackslash{}n}\StringTok{"}\NormalTok{) }\CommentTok{#add lines between datasets \}#end the bootstrap for loop}
\NormalTok{  \}}\CommentTok{#end the bootstrap for loop}
\end{Highlighting}
\end{Shaded}

\begin{verbatim}
##      PID BMI SEX AGE
## 6      6  18   0  40
## 8      8  26   1  59
## 2      2  27   0  57
## 10    10  20   0  42
## 3      3  31   1  66
## 10.1  10  20   0  42
## 9      9  34   1  65
## 3.1    3  31   1  66
## 1      1  22   1  45
## 4      4  24   1  49
##  
##  
##     PID BMI SEX AGE
## 6     6  18   0  40
## 6.1   6  18   0  40
## 10   10  20   0  42
## 8     8  26   1  59
## 7     7  21   0  65
## 8.1   8  26   1  59
## 1     1  22   1  45
## 1.1   1  22   1  45
## 6.2   6  18   0  40
## 5     5  23   0  33
##  
##  
##     PID BMI SEX AGE
## 7     7  21   0  65
## 9     9  34   1  65
## 3     3  31   1  66
## 7.1   7  21   0  65
## 3.1   3  31   1  66
## 9.1   9  34   1  65
## 8     8  26   1  59
## 9.2   9  34   1  65
## 2     2  27   0  57
## 8.1   8  26   1  59
##  
## 
\end{verbatim}

A close look at these bootstrap samples will indicate what is meant by
creating many potential studies that share the same characteristics. The
subjects in every bootstrap sample are the same as those in the data,
though some may not appear and others may appear multiple times. All
information about the subject is preserved (information across columns
is not scrambled), and the number of subjects in each bootstrap sample
is the same as the number in the original dataset. This was done by
using sampling with replacement of the subjects' IDs using the same
probability. Sampling with replacement can be conceptualized as follows:

1 consider a lottery machine that contains identical balls that have the
name, or IDs, of n subjects; 2 sample one ball, record the ID and place
the ball back into the lottery machine; 3 repeat the experiment n times.


\end{document}
